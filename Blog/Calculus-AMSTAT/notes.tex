Statistics can be introduced best when students have a solid grounding in calculus.  This statement, which would have been mainstream 50 years ago, is now controversial and provocative, even goading.  I write it, and believe it, even though there is considerable experience to the contrary and knowing that conventional wisdom is in the other camp.

For at least twenty years, there has been a strong ``reform'' movement in statistics education, an improvement in pedagogy based on an improved understanding of how students perceive statistical concepts and stripping away mathematical formalism that's not strictly needed, including integration and differentiation, the hallmarks of calculus.  This has made statistics more accessible.  

Perhaps the most visible sign of success is the rapid growth of Advanced Placement statistics in high-school.(ref)  Although there is are important and legitimate criticisms of the AP curriculum and how it connects with more advanced statistics (ref Xiao-Li Meng), the success of the AP program is inspiring and a model to be emulated, essentially bootstrapping itself into high schools by providing training opportunities for high-school teachers who often had little or no statistics education themselves.  Many students find the AP statistics course attractive because they see it as useful.  Often it is seen as an alternative to AP calculus.  

A decade ago, the Mathematical Association of America Committee on the Undergraduate Program in Mathematics, undertook a study of ``partner'' disciplines and the mathematics curriculum can serve them.  The results were published in the CRAFTY (Curriculum Reform in the First Two Years) reports, a major point of which, across partner disciplines, was that STEM students broadly be taught statistics course without a calculus pre-requisite.

Calculus instruction saw a similar reform movement.(Ref)  The outcome has been quite different than in statistics.  A charitable interpretation is that there has been a dialectic; another is that there has been a civil war. Remarkably the partner disciplines in CRAFTY strongly embraced the reform style of calculus teaching.  They assumed that this was already being done were unaware that there might be opposition.

Readers of this journal, as well as mathematics instructors generally, are an example of survival bias.  We, by and large, enjoyed calculus and find ways to use what we learned.  We read algebraic notation fluently.  The same can hardly be said for college graduates as a body.  Nationally, the most heavily enrolled mathematics course at the college level is ``college algebra,'' a pre-calculus course designed in almost all cases to lead to calculus, but with a success percentage in the single digits. That the long road to calculus through algebra and trigonometry causes a bottleneck is well known to mathematics educators.(Ref)

Even among those reaching the calculus level,  there a high attrition rate. The half-life of a student in the university-level mathematics curriculum is one course; calculus is a filter.  Although once it made sense to use the filter to allocate scarce educational resources, it has become a problem as the economy becomes more and more technical.

Even when the calculus destination is reached, it's not clear what might be the value.  A student who takes a year learning techniques for symbolic differentiation and integration of functions of a single variable, along with definitions of limit and techniques for the analysis of sequences and series, by and large learns materials that have been forgotten by instructors in the partner disciplines.  If they use modern computing at all, it will typically be a software package or hand-held calculator that isn't much used outside of mathematics instructions.  

Thus the sensible attempts to improve the situation by moving away from calculus.  The Statway [ref] project has demonstrated initial success in remedial mathematics.  Indeed, the growth of non-calculus statistics courses can be seen as a reasonable response to the problem, giving students useful quantitative skills without imposing the calculus filter.

So, yes, let's continue to support the remarkable achievements realized by offering statistics without a calculus pre-requisite.  Yet let's not forget what a calculus pre-requisite can do to enhance an understanding of statistics.

I do not mean calculus the way it has been taught for centuries.  The algebraic ``rule'' approach to calculus dates from the late 1600s, starting with de l'Hospital's textbook.  The real-analysis approach is more recent, with roots in the mid-1800s.  

Calculus instruction is a failure.  The calculus that's taught is almost entirely irrelevant to statistical thinking.  Statistics is a valuable topic, accessible to students even without calculus, while calculus stands as a barrier to many students.  A system which requires all students to study calculus before undertaking statistics would fail to serve a large group: arguably the majority.

Yet even acknowledging this, I believe that statistics can be introduced best when students have a solid grounding in calculus.  Care must be taken, though, in thinking about the meaning of ``introduced'' and, especially, ``calculus.''  

First, I'm talking about a university-level introduction to calculus.  There's great value in teaching statistics much earlier, in the middle- and secondary-school curriculum as outlined in the Common Core, or in primary school as pioneered by our colleagues in New Zealand. (ref) 

Second, I'm talking about calculus as a tool for understanding relationships and modeling relationships with functions, not the algebraic ``rules'' introduced in the late 1600s that are at the core of most introductory calculus courses nor the sampling of real analysis, rooted in the mid-1800s that is used to resolve esoteric ambiguities about continuity and limit.

For introductory statistics, the calculus concepts of approximation and partial rates of change are hugely important.  This is not the notion of approximation represented by Taylor series, based on the idea of matching derivatives of a function at a single point and concerned with the construction of more easily manipulated polynomial forms and their convergence to the ``true'' function.  Instead, it's the idea that there are relatively simple, general purpose functional frameworks for portraying relationships and the notion, dating from Gauss, that the approximation can be shaped and evaluated by discrete and realistic data.

Calculus should not be rules like $x^2 \rightarrow 2 x$ for translating a common-sense notion like ``slope'' into algebraic form.  It should be a tool for taking apart complex relationships.  Foremost, a first calculus course is an opportunity to introduce the idea of functions of multiple variables; giving students a formalism for extending from the functions of one and two variables that are easily shown graphically to functions of more variables that cannot.  A powerful idea from calculus is that of the partial derivative --- examining the change in outcome as one input is changed while others are held constant.  This aligns with the notion in science of experimental method; studying it helps students understand that there are different ways for change to happen.  In my view, understanding what's a partial change and what's not is fundamental to thinking about covariates and causation.

Mathematicians often describe calculus as ``the mathematics of the infinitesimal.''  
The infinitesimal, infinite, and convergence are, in my view, specialized and advanced topics, not often needed in any but the most informal and intuitive way.  But this does not mean that calculus is irrelevant, just that a change of emphasis is called for in the way calculus is taught.

Instead of, and certainly before, constructing high-order polynomials \'{a} la Taylor in a single variable, students should master low-order approximations in multiple variables, not $\sin(x) = x - \frac{x^3}{3!} + \frac{x^5}{5!} - \ldots$ but $f(x,y) = a_0 + a_1 x + a2_y + a_3 x y$. With at least two inputs, you can get to the idea of covariates and how an analysis might try to hold them constant.   With just the rudiments of algebra partial and the idea of partial differentiation, you can see how an interaction term, $a_3 x y$ tells how one variable modulates the effect of the other on the output: 
$\frac{\partial f}{\partial x} = a_1 + a_3 y$. 

Calculus is a language of describing and analyzing relationships.  So is statistics.  A student who knows the calculus language will have a better understanding of the statistical one.



Calculus, as the name suggests, is fundamentally associated with computing.  It's a brilliant and widely applicable strategy: breaking systems down into small, simple bits and adding up those bits to 




Enrollment data from CBMS survey: http://www.ams.org/profession/data/cbms-survey/Chapter1DraftTables2.pdf

Table S2.  Total enrollments in thousands.

Conversation with Bill Gates: http://chronicle.com/article/A-Conversation-With-Bill-Gates/132591/?cid=at&utm_source=at&utm_medium=en

Allies or competitors?  Two of the most important subjects in the university-level mathematical sciences are calculus and statistics. In terms of overall enrollments, the topics and the various algebra and pre-calc courses that are directed to calculus, are the dominant players.  There are ways in which the topics compete, and a strong view that calculus is mainly irrelevant to statistics. I will argue that there are also strong and natural connections between the two --- that they should be allied. The alliance is most productive, and most beneficial to students, if we update our conception of calculus.

Lhopital. Introduction of stat it 20th century. Calc is an 18th century field. No wonder calc doesn't draw on stat. But why isn't calc essential to stat


Does taking calculus help students of introductory statistics?  Many instructors would say no, that students who take calculus are more mathematically mature, but the topics of calculus: limits, differentiation, integration, series, ... are not relevant or informative when it comes to introductory statistics.  Almost uniformly, textbooks don't use the calculus operations in presenting introductory statistics.

I think this view is wrong-headed: concepts from calculus are important to using and interpreting statistical methodology.  A student who knows something about calculus has a strong advantage in learning statistics.

Don't get me wrong.  The ``calculus'' that gets taught in many university-level courses and in the AP curriculum has little to do with introductory statistics.  But we're teaching the wrong calculus.  The right calculus is closely connected to statistics and illuminates statistical thinking.

The first calculus textbook was published in 1699, by the Marquis de l'Hopital.  The topics in l'Hopital's book --- the ``rules'' and notation --- are echoed in today's textbooks.  Don't think that's because calculus has not changed, that calculus is immutable and eternal.  It's just that calculus education is mainly based on the same technologies and techniques that were available to de l'Hopital and his near contemporaries.

Take, for instance, Taylor series.  Geoffrey Taylor was a contemporary of Isaac Newton.   He must have been very excited to find an application for derivatives ....  They are not used so much except for algebraic purposes: low-order polynomial approximations to functions whose derivatives are known.  There are other ways to make polynomial approximations, for instance least squares.   Why the emphasis on Taylor and not on Gauss?  Taylor polynomials offer many opportunities for exam questions about derivatives.  Least squares, by and large, requires a computer.

Mathematicians will often say, ``that function doesn't have an anti-derivative.''  What they mean is that there isn't an anti-derivative in a simple, algebraic form.



related rates, partial fractions, trigonometric substitution, l'Hopital's rule ... not useful.  Technology in calculus education often consists of replacing student labor in symbolic calculations with computer algebra, using special software that is never encountered in any other field.



Those who follow trends in secondary and post-secondary education know that there has emerged something of a conflict between statistics and calculus.  High-school AP statistics is often presented as an alternative to calculus for weaker students.  College biology departments often allow students to fulfill a math requirement with either calculus or statistics.  The pre-medical requirements are shifting --- helpfully --- to put a greater emphasis on statistics and epidemiology and psychology and removing a requirement for calculus.

It's good for statistics educators to offer access to statistics without requiring calculus.  


Trouble in Calculus Education 

Every five years, the Conference Board of Mathematical Sciences undertakes a survey of US undergraduate enrollment in mathematics.  The graphic shows some of the 2010 and earlier CBMS data on enrollment by ``level'' of course: pre-calculus, calculus, statistics, etc.

GRAPHIC HERE

Many statistics professionals are surprised to see that the highest enrollment undergraduate course is at the ``pre-college'' level.   Next highest enrollment is pre-calculus, then calculus.  

The pattern displayed might suggest that students move from pre-college to pre-calculus to calculus with some attrition along the way.  But that's misleading.  This is a cross-sectional survey.  Few of the pre-calculus students will go on to calculus; most of the undergraduate calculus enrollments are coming from a high-school preparation, not a college pre-calc or pre-college level.  Attrition in calculus is high, so that even those few students who run the pre-calc gauntlet will ever reach a point where they can employ the tools of calculus.  In fact, there's considerable skepticism that the ``tools of calculus'' are useful for much at all besides physics or engineering, even among the few who complete a calculus course.

It may seem the right thing, then, for statistics educators to by-pass calculus.  Indeed, other fields agree: reports such as the MAA's Curricular Reform In the First Two Years report call for non-calculus based statistics.  A mainstream introductory statistics course will not have a calculus pre-requisite.  When there is a calculus pre-requisite, it's often regarded by faculty as representing ``mathematical maturity'' rather than the need to know something about calculus.

Given the broad importance of statistics, the failure of the mathematics education system to provide a calculus program what works for a large proportion of students, it's appropriate to offer access to statistics courses without the gateway of calculus.  



