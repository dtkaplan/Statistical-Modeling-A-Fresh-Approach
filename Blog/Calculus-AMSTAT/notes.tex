

Enrollment data from CBMS survey: http://www.ams.org/profession/data/cbms-survey/Chapter1DraftTables2.pdf

Table S2.  Total enrollments in thousands.

Conversation with Bill Gates: http://chronicle.com/article/A-Conversation-With-Bill-Gates/132591/?cid=at&utm_source=at&utm_medium=en

Allies or competitors?  Two of the most important subjects in the university-level mathematical sciences are calculus and statistics. In terms of overall enrollments, the topics and the various algebra and pre-calc courses that are directed to calculus, are the dominant players.  There are ways in which the topics compete, and a strong view that calculus is mainly irrelevant to statistics. I will argue that there are also strong and natural connections between the two --- that they should be allied. The alliance is most productive, and most beneficial to students, if we update our conception of calculus.

Lhopital. Introduction of stat it 20th century. Calc is an 18th century field. No wonder calc doesn't draw on stat. But why isn't calc essential to stat


Does taking calculus help students of introductory statistics?  Many instructors would say no, that students who take calculus are more mathematically mature, but the topics of calculus: limits, differentiation, integration, series, ... are not relevant or informative when it comes to introductory statistics.  Almost uniformly, textbooks don't use the calculus operations in presenting introductory statistics.

I think this view is wrong-headed: concepts from calculus are important to using and interpreting statistical methodology.  A student who knows something about calculus has a strong advantage in learning statistics.

Don't get me wrong.  The ``calculus'' that gets taught in many university-level courses and in the AP curriculum has little to do with introductory statistics.  But we're teaching the wrong calculus.  The right calculus is closely connected to statistics and illuminates statistical thinking.

The first calculus textbook was published in 1699, by the Marquis de l'Hopital.  The topics in l'Hopital's book --- the ``rules'' and notation --- are echoed in today's textbooks.  Don't think that's because calculus has not changed, that calculus is immutable and eternal.  It's just that calculus education is mainly based on the same technologies and techniques that were available to de l'Hopital and his near contemporaries.

Take, for instance, Taylor series.  Geoffrey Taylor was a contemporary of Isaac Newton.   He must have been very excited to find an application for derivatives ....  They are not used so much except for algebraic purposes: low-order polynomial approximations to functions whose derivatives are known.  There are other ways to make polynomial approximations, for instance least squares.   Why the emphasis on Taylor and not on Gauss?  Taylor polynomials offer many opportunities for exam questions about derivatives.  Least squares, by and large, requires a computer.

Mathematicians will often say, ``that function doesn't have an anti-derivative.''  What they mean is that there isn't an anti-derivative in a simple, algebraic form.



related rates, partial fractions, trigonometric substitution, l'Hopital's rule ... not useful.  Technology in calculus education often consists of replacing student labor in symbolic calculations with computer algebra, using special software that is never encountered in any other field.



Those who follow trends in secondary and post-secondary education know that there has emerged something of a conflict between statistics and calculus.  High-school AP statistics is often presented as an alternative to calculus for weaker students.  College biology departments often allow students to fulfill a math requirement with either calculus or statistics.  The pre-medical requirements are shifting --- helpfully --- to put a greater emphasis on statistics and epidemiology and psychology and removing a requirement for calculus.

It's good for statistics educators to offer access to statistics without requiring calculus.  


Trouble in Calculus Education 

Every five years, the Conference Board of Mathematical Sciences undertakes a survey of US undergraduate enrollment in mathematics.  The graphic shows some of the 2010 and earlier CBMS data on enrollment by ``level'' of course: pre-calculus, calculus, statistics, etc.

GRAPHIC HERE

Many statistics professionals are surprised to see that the highest enrollment undergraduate course is at the ``pre-college'' level.   Next highest enrollment is pre-calculus, then calculus.  

The pattern displayed might suggest that students move from pre-college to pre-calculus to calculus with some attrition along the way.  But that's misleading.  This is a cross-sectional survey.  Few of the pre-calculus students will go on to calculus; most of the undergraduate calculus enrollments are coming from a high-school preparation, not a college pre-calc or pre-college level.  Attrition in calculus is high, so that even those few students who run the pre-calc gauntlet will ever reach a point where they can employ the tools of calculus.  In fact, there's considerable skepticism that the ``tools of calculus'' are useful for much at all besides physics or engineering, even among the few who complete a calculus course.

It may seem the right thing, then, for statistics educators to by-pass calculus.  Indeed, other fields agree: reports such as the MAA's Curricular Reform In the First Two Years report call for non-calculus based statistics.  A mainstream introductory statistics course will not have a calculus pre-requisite.  When there is a calculus pre-requisite, it's often regarded by faculty as representing ``mathematical maturity'' rather than the need to know something about calculus.

Given the broad importance of statistics, the failure of the mathematics education system to provide a calculus program what works for a large proportion of students, it's appropriate to offer access to statistics courses without the gateway of calculus.  



